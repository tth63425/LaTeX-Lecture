%最初の文書ができたところで,コマンドの意味を考えてみましょう!

%まずこの行の一番左の%は「コメントアウト」といいます.同じ行の%以降はPDFとして出力されません.


\documentclass{jsarticle}
%上の行はDocumant Class(文書タイプ)といって,文書が日本語か英語か?レポートか本か?などを指定します.
%PDFに書かれるわけではないですが,テキストの一番上に必ず書かないといけません.基本的には\documentclass{jsarticle}のままで大丈夫です.


\usepackage[dvipdfmx]{graphicx}
%この行はパッケージを使うためのコマンドです.
%パッケージとは,LaTeXに標準で入っていない機能を使いたいときに使う「道具」のようなものです.なので,この行はなくても大丈夫ですが,書く時には必ず\documentclass{}と(下にでてくる)\begin{document}に書くという決まりがあります.
%ちなみに上のパッケージは画像をPDF中に入れるためのパッケージを使いたいです,という意味です.
%パッケージをここで入れておかないと使えないコマンドも多数あります.エラーメッセージが出た場合に必要なパッケージが入っているか確認するのもよいでしょう.


          \title{『計量経済学の第一歩』 第3章}
          %上のコマンドでタイトル指定します.{}の中にタイトルを書きましょう.

          \author{佐藤 彰}
          %このコマンドで著者を指定します.{}の中を自分の名前に変えておきましょう.

          \date{\today}
          %このコマンドで日付を指定します.{}の中に\todayと書いておくと自動で今日の日付が入ります.
          % Q. レジュメ直前に書いたのばれると恥ずかしいから日付変えたいです...?
          % A. 簡単です!{}のなかに2019年4月18日と書けばいいのです!


\begin{document}
%ここから文章(本文)が始まりますよ,とLaTeXに教えるためのコマンドです.


\maketitle
%タイトル(上のへこんでいる部分です)を出力するコマンドです.


\section{物事の起こりやすさを表すツールとしての「確率」}
%章の名前を書くためのコマンドです.{}の中に名前を書きます.番号は自動で振られます.
%ゼミのレジュメではここに節の名前を書くことになると思います.


\subsection{「確率」とは}
%節の名前を書くためのコマンドです.{}の中に名前書きます.番号は自動で振られます.
%ゼミのレジュメだったら小節の名前を書きます(あんまり使わないかも!?)


\begin{itemize}
\item 確率とは,それぞれの事象がどの程度「確からしい」のかを0から1の数字で表したもの
\item 事象とは,起こりうる結果を指している
\end{itemize}
%上のコマンドは「箇条書き」をするためのコマンドです.注意としては,\itemのあとに必ず半角のスペースを入れましょう.

\subsection{確率のお約束}
\textgt{排反事象}とは,一方が起これば他方は起こらないという関係にある事象のこと.
%ある文字を太字にしたい場合は\textgt{}の{}の中に太字にしたい文字を入れます.

\section{段落はどうやって作ればいい?}
政府は今年3月14日付で国文、漢文、日本史、東洋史の学者に元号の考案を委嘱。1日の決定過程では、中西氏が考案した「令和」を含む6案を有識者9人による「元号に関する懇談会」などに提示した。

政府が公表した議事概要によると、懇談会の場では「令和」について、「我が国がもっている素晴らしい洗練された文化を象徴している」など賛同する意見が相次いだとされる。

懇談会などの詳細な議事録は別途作成して、「令和」の次の元号が決まった後に国立公文書館に移管して公開する方針だ。
(「朝日新聞デジタル」2019年4月20日)
%ここで見てほしいのはLaTeXでどのように段落が記述されているかです.LaTeXでは,段落を作るためには「先頭にスペースは入れず」,「1行分開ける」必要があります.
%エディタで空白行なしで改行しただけでは何もPDFには反映されません(「朝日新聞」の行参照)



\end{document}
